% Options for packages loaded elsewhere
\PassOptionsToPackage{unicode}{hyperref}
\PassOptionsToPackage{hyphens}{url}
%
\documentclass[
]{article}
\usepackage{amsmath,amssymb}
\usepackage{lmodern}
\usepackage{iftex}
\ifPDFTeX
  \usepackage[T1]{fontenc}
  \usepackage[utf8]{inputenc}
  \usepackage{textcomp} % provide euro and other symbols
\else % if luatex or xetex
  \usepackage{unicode-math}
  \defaultfontfeatures{Scale=MatchLowercase}
  \defaultfontfeatures[\rmfamily]{Ligatures=TeX,Scale=1}
\fi
% Use upquote if available, for straight quotes in verbatim environments
\IfFileExists{upquote.sty}{\usepackage{upquote}}{}
\IfFileExists{microtype.sty}{% use microtype if available
  \usepackage[]{microtype}
  \UseMicrotypeSet[protrusion]{basicmath} % disable protrusion for tt fonts
}{}
\makeatletter
\@ifundefined{KOMAClassName}{% if non-KOMA class
  \IfFileExists{parskip.sty}{%
    \usepackage{parskip}
  }{% else
    \setlength{\parindent}{0pt}
    \setlength{\parskip}{6pt plus 2pt minus 1pt}}
}{% if KOMA class
  \KOMAoptions{parskip=half}}
\makeatother
\usepackage{xcolor}
\usepackage[margin=1in]{geometry}
\usepackage{color}
\usepackage{fancyvrb}
\newcommand{\VerbBar}{|}
\newcommand{\VERB}{\Verb[commandchars=\\\{\}]}
\DefineVerbatimEnvironment{Highlighting}{Verbatim}{commandchars=\\\{\}}
% Add ',fontsize=\small' for more characters per line
\usepackage{framed}
\definecolor{shadecolor}{RGB}{248,248,248}
\newenvironment{Shaded}{\begin{snugshade}}{\end{snugshade}}
\newcommand{\AlertTok}[1]{\textcolor[rgb]{0.94,0.16,0.16}{#1}}
\newcommand{\AnnotationTok}[1]{\textcolor[rgb]{0.56,0.35,0.01}{\textbf{\textit{#1}}}}
\newcommand{\AttributeTok}[1]{\textcolor[rgb]{0.77,0.63,0.00}{#1}}
\newcommand{\BaseNTok}[1]{\textcolor[rgb]{0.00,0.00,0.81}{#1}}
\newcommand{\BuiltInTok}[1]{#1}
\newcommand{\CharTok}[1]{\textcolor[rgb]{0.31,0.60,0.02}{#1}}
\newcommand{\CommentTok}[1]{\textcolor[rgb]{0.56,0.35,0.01}{\textit{#1}}}
\newcommand{\CommentVarTok}[1]{\textcolor[rgb]{0.56,0.35,0.01}{\textbf{\textit{#1}}}}
\newcommand{\ConstantTok}[1]{\textcolor[rgb]{0.00,0.00,0.00}{#1}}
\newcommand{\ControlFlowTok}[1]{\textcolor[rgb]{0.13,0.29,0.53}{\textbf{#1}}}
\newcommand{\DataTypeTok}[1]{\textcolor[rgb]{0.13,0.29,0.53}{#1}}
\newcommand{\DecValTok}[1]{\textcolor[rgb]{0.00,0.00,0.81}{#1}}
\newcommand{\DocumentationTok}[1]{\textcolor[rgb]{0.56,0.35,0.01}{\textbf{\textit{#1}}}}
\newcommand{\ErrorTok}[1]{\textcolor[rgb]{0.64,0.00,0.00}{\textbf{#1}}}
\newcommand{\ExtensionTok}[1]{#1}
\newcommand{\FloatTok}[1]{\textcolor[rgb]{0.00,0.00,0.81}{#1}}
\newcommand{\FunctionTok}[1]{\textcolor[rgb]{0.00,0.00,0.00}{#1}}
\newcommand{\ImportTok}[1]{#1}
\newcommand{\InformationTok}[1]{\textcolor[rgb]{0.56,0.35,0.01}{\textbf{\textit{#1}}}}
\newcommand{\KeywordTok}[1]{\textcolor[rgb]{0.13,0.29,0.53}{\textbf{#1}}}
\newcommand{\NormalTok}[1]{#1}
\newcommand{\OperatorTok}[1]{\textcolor[rgb]{0.81,0.36,0.00}{\textbf{#1}}}
\newcommand{\OtherTok}[1]{\textcolor[rgb]{0.56,0.35,0.01}{#1}}
\newcommand{\PreprocessorTok}[1]{\textcolor[rgb]{0.56,0.35,0.01}{\textit{#1}}}
\newcommand{\RegionMarkerTok}[1]{#1}
\newcommand{\SpecialCharTok}[1]{\textcolor[rgb]{0.00,0.00,0.00}{#1}}
\newcommand{\SpecialStringTok}[1]{\textcolor[rgb]{0.31,0.60,0.02}{#1}}
\newcommand{\StringTok}[1]{\textcolor[rgb]{0.31,0.60,0.02}{#1}}
\newcommand{\VariableTok}[1]{\textcolor[rgb]{0.00,0.00,0.00}{#1}}
\newcommand{\VerbatimStringTok}[1]{\textcolor[rgb]{0.31,0.60,0.02}{#1}}
\newcommand{\WarningTok}[1]{\textcolor[rgb]{0.56,0.35,0.01}{\textbf{\textit{#1}}}}
\usepackage{graphicx}
\makeatletter
\def\maxwidth{\ifdim\Gin@nat@width>\linewidth\linewidth\else\Gin@nat@width\fi}
\def\maxheight{\ifdim\Gin@nat@height>\textheight\textheight\else\Gin@nat@height\fi}
\makeatother
% Scale images if necessary, so that they will not overflow the page
% margins by default, and it is still possible to overwrite the defaults
% using explicit options in \includegraphics[width, height, ...]{}
\setkeys{Gin}{width=\maxwidth,height=\maxheight,keepaspectratio}
% Set default figure placement to htbp
\makeatletter
\def\fps@figure{htbp}
\makeatother
\setlength{\emergencystretch}{3em} % prevent overfull lines
\providecommand{\tightlist}{%
  \setlength{\itemsep}{0pt}\setlength{\parskip}{0pt}}
\setcounter{secnumdepth}{-\maxdimen} % remove section numbering
\ifLuaTeX
  \usepackage{selnolig}  % disable illegal ligatures
\fi
\IfFileExists{bookmark.sty}{\usepackage{bookmark}}{\usepackage{hyperref}}
\IfFileExists{xurl.sty}{\usepackage{xurl}}{} % add URL line breaks if available
\urlstyle{same} % disable monospaced font for URLs
\hypersetup{
  pdftitle={FE515 2022A Midterm},
  pdfauthor={Yufu Liao},
  hidelinks,
  pdfcreator={LaTeX via pandoc}}

\title{FE515 2022A Midterm}
\author{Yufu Liao}
\date{03/22/2023}

\begin{document}
\maketitle

\hypertarget{question-1-50-points}{%
\section{Question 1: (50 points)}\label{question-1-50-points}}

\hypertarget{section}{%
\subsection{1.1}\label{section}}

Download daily equity data of JPM and WFC (2012-01-01 to 2023-01-01)

\begin{Shaded}
\begin{Highlighting}[]
\CommentTok{\#install.packages(\textquotesingle{}quantmod\textquotesingle{})}
\FunctionTok{library}\NormalTok{(quantmod)}
\end{Highlighting}
\end{Shaded}

\begin{verbatim}
## Loading required package: xts
\end{verbatim}

\begin{verbatim}
## Loading required package: zoo
\end{verbatim}

\begin{verbatim}
## 
## Attaching package: 'zoo'
\end{verbatim}

\begin{verbatim}
## The following objects are masked from 'package:base':
## 
##     as.Date, as.Date.numeric
\end{verbatim}

\begin{verbatim}
## Loading required package: TTR
\end{verbatim}

\begin{verbatim}
## Registered S3 method overwritten by 'quantmod':
##   method            from
##   as.zoo.data.frame zoo
\end{verbatim}

\begin{Shaded}
\begin{Highlighting}[]
\FunctionTok{getSymbols}\NormalTok{(}\AttributeTok{Symbols =} \StringTok{"JPM"}\NormalTok{, }\AttributeTok{from =} \StringTok{"2012{-}01{-}01"}\NormalTok{, }\AttributeTok{to =} \StringTok{\textquotesingle{}2023{-}01{-}01\textquotesingle{}}\NormalTok{)}
\end{Highlighting}
\end{Shaded}

\begin{verbatim}
## Warning in strptime(xx, ff, tz = "GMT"): unable to identify current timezone 'T':
## please set environment variable 'TZ'
\end{verbatim}

\begin{verbatim}
## [1] "JPM"
\end{verbatim}

\begin{Shaded}
\begin{Highlighting}[]
\NormalTok{JPM }\OtherTok{\textless{}{-}} \FunctionTok{data.frame}\NormalTok{(JPM)}
\FunctionTok{head}\NormalTok{(JPM)}
\end{Highlighting}
\end{Shaded}

\begin{verbatim}
##            JPM.Open JPM.High JPM.Low JPM.Close JPM.Volume JPM.Adjusted
## 2012-01-03    34.06    35.19   34.01     34.98   44102800     25.54859
## 2012-01-04    34.44    35.15   34.33     34.95   36571200     25.71043
## 2012-01-05    34.71    35.92   34.40     35.68   38381400     26.24744
## 2012-01-06    35.69    35.77   35.14     35.36   33160600     26.01205
## 2012-01-09    35.44    35.68   34.99     35.30   23001800     25.96790
## 2012-01-10    36.07    36.35   35.76     36.05   35972800     26.51963
\end{verbatim}

\begin{Shaded}
\begin{Highlighting}[]
\FunctionTok{getSymbols}\NormalTok{(}\AttributeTok{Symbols =} \StringTok{"WFC"}\NormalTok{, }\AttributeTok{from =} \StringTok{"2012{-}01{-}01"}\NormalTok{, }\AttributeTok{to =} \StringTok{\textquotesingle{}2023{-}01{-}01\textquotesingle{}}\NormalTok{)}
\end{Highlighting}
\end{Shaded}

\begin{verbatim}
## [1] "WFC"
\end{verbatim}

\begin{Shaded}
\begin{Highlighting}[]
\NormalTok{WFC }\OtherTok{\textless{}{-}} \FunctionTok{data.frame}\NormalTok{(WFC)}
\FunctionTok{head}\NormalTok{(WFC)}
\end{Highlighting}
\end{Shaded}

\begin{verbatim}
##            WFC.Open WFC.High WFC.Low WFC.Close WFC.Volume WFC.Adjusted
## 2012-01-03    27.94    28.52   27.94     28.43   40071200     20.65028
## 2012-01-04    28.34    28.69   28.04     28.56   27519200     20.74472
## 2012-01-05    28.50    29.58   28.25     29.02   48435100     21.07883
## 2012-01-06    28.84    29.08   28.46     28.94   32303500     21.02073
## 2012-01-09    29.15    29.38   29.00     29.30   25720100     21.28221
## 2012-01-10    29.74    29.80   29.18     29.41   29860100     21.36211
\end{verbatim}

\hypertarget{section-1}{%
\subsection{1.2}\label{section-1}}

Calculate both the daily log return and weekly log return for each
stock.

\begin{Shaded}
\begin{Highlighting}[]
\NormalTok{JPM.log.daily.return }\OtherTok{\textless{}{-}} \FunctionTok{diff}\NormalTok{(}\FunctionTok{log}\NormalTok{(JPM}\SpecialCharTok{$}\NormalTok{JPM.Adjusted))}
\NormalTok{JPM.log.weekly.return }\OtherTok{\textless{}{-}} \FunctionTok{periodReturn}\NormalTok{(JPM, }\AttributeTok{type =} \StringTok{\textquotesingle{}log\textquotesingle{}}\NormalTok{, }\AttributeTok{period =} \StringTok{\textquotesingle{}weekly\textquotesingle{}}\NormalTok{)}

\NormalTok{WFC.log.daily.return }\OtherTok{\textless{}{-}} \FunctionTok{diff}\NormalTok{(}\FunctionTok{log}\NormalTok{(WFC}\SpecialCharTok{$}\NormalTok{WFC.Adjusted))}
\NormalTok{WFC.log.weekly.return }\OtherTok{\textless{}{-}} \FunctionTok{periodReturn}\NormalTok{(WFC, }\AttributeTok{type =} \StringTok{\textquotesingle{}log\textquotesingle{}}\NormalTok{, }\AttributeTok{period =} \StringTok{\textquotesingle{}weekly\textquotesingle{}}\NormalTok{)}
\end{Highlighting}
\end{Shaded}

\hypertarget{section-2}{%
\subsection{1.3}\label{section-2}}

Visualize the distribution of these log returns using hist() function.
Use par() function to put the four histogram together into one single
graph, where each histogram is an individual subplot.

\begin{Shaded}
\begin{Highlighting}[]
\FunctionTok{par}\NormalTok{(}\AttributeTok{mfrow =} \FunctionTok{c}\NormalTok{(}\DecValTok{2}\NormalTok{, }\DecValTok{2}\NormalTok{))}
\FunctionTok{hist}\NormalTok{(JPM.log.daily.return)}
\FunctionTok{hist}\NormalTok{(JPM.log.weekly.return)}
\FunctionTok{hist}\NormalTok{(WFC.log.daily.return)}
\FunctionTok{hist}\NormalTok{(WFC.log.weekly.return)}
\end{Highlighting}
\end{Shaded}

\includegraphics{YufuLiao_Midterm_files/figure-latex/unnamed-chunk-3-1.pdf}

\hypertarget{section-3}{%
\subsection{1.4}\label{section-3}}

Calculate the first four moments, i.e.~mean, variance, skewness and
kurtosis, for each stock. Store the calculate result in a data frame and
report the result in a table.

\begin{Shaded}
\begin{Highlighting}[]
\FunctionTok{library}\NormalTok{(}\StringTok{\textquotesingle{}moments\textquotesingle{}}\NormalTok{)}

\NormalTok{jpm\_moments }\OtherTok{\textless{}{-}} \FunctionTok{c}\NormalTok{(}\FunctionTok{mean}\NormalTok{(JPM.log.daily.return), }\FunctionTok{var}\NormalTok{(JPM.log.daily.return), moments}\SpecialCharTok{::}\FunctionTok{skewness}\NormalTok{(JPM.log.daily.return), moments}\SpecialCharTok{::}\FunctionTok{kurtosis}\NormalTok{(JPM.log.daily.return))}

\CommentTok{\# Calculate the first four moments for WFC daily log returns}
\NormalTok{wfc\_moments }\OtherTok{\textless{}{-}} \FunctionTok{c}\NormalTok{(}\FunctionTok{mean}\NormalTok{(WFC.log.daily.return), }\FunctionTok{var}\NormalTok{(WFC.log.daily.return), moments}\SpecialCharTok{::}\FunctionTok{skewness}\NormalTok{(WFC.log.daily.return), moments}\SpecialCharTok{::}\FunctionTok{kurtosis}\NormalTok{(WFC.log.daily.return))}

\CommentTok{\# Combine the results into a data frame}
\NormalTok{moments\_df }\OtherTok{\textless{}{-}} \FunctionTok{data.frame}\NormalTok{(}\AttributeTok{Statistic =} \FunctionTok{c}\NormalTok{(}\StringTok{"Mean"}\NormalTok{, }\StringTok{"Variance"}\NormalTok{, }\StringTok{"Skewness"}\NormalTok{, }\StringTok{"Kurtosis"}\NormalTok{), }\AttributeTok{JPM =}\NormalTok{ jpm\_moments, }\AttributeTok{WFC =}\NormalTok{ wfc\_moments)}

\CommentTok{\# Print the data frame}
\FunctionTok{print}\NormalTok{(moments\_df)}
\end{Highlighting}
\end{Shaded}

\begin{verbatim}
##   Statistic           JPM           WFC
## 1      Mean  0.0005965465  0.0002481145
## 2  Variance  0.0002875018  0.0003325385
## 3  Skewness -0.1100152687 -0.3475361071
## 4  Kurtosis 15.7443325384 14.1951091034
\end{verbatim}

\hypertarget{section-4}{%
\subsection{1.5}\label{section-4}}

Draw a scatter plot of JPM daily return against WFC daily return.
(i.e.~WFC return on x-axis and JPM return on y-axis)

\begin{Shaded}
\begin{Highlighting}[]
\NormalTok{returns\_df }\OtherTok{\textless{}{-}} \FunctionTok{data.frame}\NormalTok{(}\AttributeTok{JPM =}\NormalTok{ JPM.log.daily.return, }\AttributeTok{WFC =}\NormalTok{ WFC.log.daily.return)}

\FunctionTok{plot}\NormalTok{(JPM }\SpecialCharTok{\textasciitilde{}}\NormalTok{ WFC, }\AttributeTok{data =}\NormalTok{ returns\_df, }\AttributeTok{xlab =} \StringTok{"WFC Daily Log Return"}\NormalTok{, }\AttributeTok{ylab =} \StringTok{"JPM Daily Log Return"}\NormalTok{,}
     \AttributeTok{main =} \StringTok{"Scatter Plot of JPM Daily Return vs WFC Daily Return"}\NormalTok{)}
\end{Highlighting}
\end{Shaded}

\includegraphics{YufuLiao_Midterm_files/figure-latex/unnamed-chunk-5-1.pdf}

\hypertarget{section-5}{%
\subsection{1.6}\label{section-5}}

Build a simple linear regression model using the WFC daily return as
explanatory variable and the JPM daily return as response variable.
Report the fitted model using summary( ) function.

\begin{Shaded}
\begin{Highlighting}[]
\NormalTok{model }\OtherTok{\textless{}{-}} \FunctionTok{lm}\NormalTok{(JPM }\SpecialCharTok{\textasciitilde{}}\NormalTok{ WFC, }\AttributeTok{data =}\NormalTok{ returns\_df)}

\FunctionTok{summary}\NormalTok{(model)}
\end{Highlighting}
\end{Shaded}

\begin{verbatim}
## 
## Call:
## lm(formula = JPM ~ WFC, data = returns_df)
## 
## Residuals:
##       Min        1Q    Median        3Q       Max 
## -0.100468 -0.005141 -0.000134  0.005129  0.070679 
## 
## Coefficients:
##              Estimate Std. Error t value Pr(>|t|)    
## (Intercept) 0.0004122  0.0001938   2.127   0.0335 *  
## WFC         0.7430600  0.0106299  69.903   <2e-16 ***
## ---
## Signif. codes:  0 '***' 0.001 '**' 0.01 '*' 0.05 '.' 0.1 ' ' 1
## 
## Residual standard error: 0.01019 on 2765 degrees of freedom
## Multiple R-squared:  0.6386, Adjusted R-squared:  0.6385 
## F-statistic:  4886 on 1 and 2765 DF,  p-value: < 2.2e-16
\end{verbatim}

\hypertarget{section-6}{%
\subsection{1.7}\label{section-6}}

Draw a regression line on the scatter plot using the fitted model above.
Make sure use a different color to draw the regression line.

\begin{Shaded}
\begin{Highlighting}[]
\FunctionTok{plot}\NormalTok{(JPM }\SpecialCharTok{\textasciitilde{}}\NormalTok{ WFC, }\AttributeTok{data =}\NormalTok{ returns\_df, }\AttributeTok{xlab =} \StringTok{"WFC Daily Log Return"}\NormalTok{, }\AttributeTok{ylab =} \StringTok{"JPM Daily Log Return"}\NormalTok{,}
     \AttributeTok{main =} \StringTok{"Scatter Plot of JPM Daily Return vs WFC Daily Return"}\NormalTok{)}

\FunctionTok{abline}\NormalTok{(model, }\AttributeTok{col =} \StringTok{"red"}\NormalTok{)}
\end{Highlighting}
\end{Shaded}

\includegraphics{YufuLiao_Midterm_files/figure-latex/unnamed-chunk-7-1.pdf}

\hypertarget{question-2}{%
\section{Question 2}\label{question-2}}

\hypertarget{section-7}{%
\subsection{2.1}\label{section-7}}

Without using packages, create a function of 2 variables ``x'' and
``adjusted'' that calculates the sample skewness of ``x'' using the
formulas on Lecture 6 page 20 and page 21. When ''adjusted'' = TRUE, it
returns the adjusted skewness of ``x'', and FALSE returns the unadjusted
one.

\begin{Shaded}
\begin{Highlighting}[]
\NormalTok{fun }\OtherTok{\textless{}{-}} \ControlFlowTok{function}\NormalTok{(x, adjusted) \{}

  \ControlFlowTok{if}\NormalTok{(}\FunctionTok{length}\NormalTok{(x) }\SpecialCharTok{\textless{}=} \DecValTok{2}\NormalTok{) }
    \FunctionTok{stop}\NormalTok{(}\StringTok{"please give more than 3 elements to calculate skewness"}\NormalTok{)}

\NormalTok{  m3 }\OtherTok{\textless{}{-}} \FunctionTok{sum}\NormalTok{((x }\SpecialCharTok{{-}} \FunctionTok{mean}\NormalTok{(x))}\SpecialCharTok{\^{}}\DecValTok{3}\NormalTok{) }\SpecialCharTok{/} \FunctionTok{length}\NormalTok{(x)}
\NormalTok{  m2 }\OtherTok{\textless{}{-}} \FunctionTok{sum}\NormalTok{((x }\SpecialCharTok{{-}} \FunctionTok{mean}\NormalTok{(x))}\SpecialCharTok{\^{}}\DecValTok{2}\NormalTok{) }\SpecialCharTok{/} \FunctionTok{length}\NormalTok{(x) }
\NormalTok{  result }\OtherTok{\textless{}{-}}\NormalTok{ m3 }\SpecialCharTok{/}\NormalTok{ (m2}\SpecialCharTok{\^{}}\NormalTok{(}\DecValTok{3}\SpecialCharTok{/}\DecValTok{2}\NormalTok{))}
  \ControlFlowTok{if}\NormalTok{(adjusted) \{}
\NormalTok{    result }\OtherTok{\textless{}{-}}\NormalTok{ result }\SpecialCharTok{*} \FunctionTok{sqrt}\NormalTok{((}\FunctionTok{length}\NormalTok{(x) }\SpecialCharTok{*}\NormalTok{ (}\FunctionTok{length}\NormalTok{(x) }\SpecialCharTok{{-}} \DecValTok{1}\NormalTok{))) }\SpecialCharTok{/}\NormalTok{ (}\FunctionTok{length}\NormalTok{(x) }\SpecialCharTok{{-}} \DecValTok{2}\NormalTok{)}
\NormalTok{  \}}
  \FunctionTok{return}\NormalTok{(result)}
\NormalTok{\}}
\end{Highlighting}
\end{Shaded}

\hypertarget{section-8}{%
\subsection{2.2}\label{section-8}}

Without using packages, create a function of 2 variables ``x'' and
''adjusted'' that calculates the sample kurtosis of ``x'' using the
formulas on Lecture 6 page 20 and page 23. When ``adjusted'' = TRUE, it
returns the adjusted kurtosis of ``x'', and FALSE returns the unadjusted
one.

\begin{Shaded}
\begin{Highlighting}[]
\NormalTok{fun2 }\OtherTok{\textless{}{-}} \ControlFlowTok{function}\NormalTok{(x, adjusted) \{}
\NormalTok{  n }\OtherTok{\textless{}{-}} \FunctionTok{length}\NormalTok{(x)}
  \ControlFlowTok{if}\NormalTok{(n }\SpecialCharTok{\textless{}=} \DecValTok{3}\NormalTok{) \{}
    \FunctionTok{stop}\NormalTok{(}\StringTok{"please give more than 4 elements to calculate kurtosis"}\NormalTok{)}
\NormalTok{  \}}
\NormalTok{  m4 }\OtherTok{\textless{}{-}} \FunctionTok{sum}\NormalTok{((x }\SpecialCharTok{{-}} \FunctionTok{mean}\NormalTok{(x))}\SpecialCharTok{\^{}}\DecValTok{4}\NormalTok{) }\SpecialCharTok{/}\NormalTok{ n}
\NormalTok{  m2 }\OtherTok{\textless{}{-}} \FunctionTok{sum}\NormalTok{((x }\SpecialCharTok{{-}} \FunctionTok{mean}\NormalTok{(x))}\SpecialCharTok{\^{}}\DecValTok{2}\NormalTok{) }\SpecialCharTok{/}\NormalTok{ n}
\NormalTok{  kurtosis }\OtherTok{\textless{}{-}}\NormalTok{ (m4 }\SpecialCharTok{/}\NormalTok{ m2}\SpecialCharTok{\^{}}\DecValTok{2}\NormalTok{)}
  \ControlFlowTok{if}\NormalTok{(adjusted) \{}
\NormalTok{    kurtosis }\OtherTok{\textless{}{-}}\NormalTok{ ((n}\DecValTok{{-}1}\NormalTok{)  }\SpecialCharTok{*}\NormalTok{ ( (n}\SpecialCharTok{+}\DecValTok{1}\NormalTok{) }\SpecialCharTok{*}\NormalTok{ kurtosis }\SpecialCharTok{{-}} \DecValTok{3} \SpecialCharTok{*}\NormalTok{ (n}\DecValTok{{-}1}\NormalTok{) ) }\SpecialCharTok{/}\NormalTok{ ((n}\DecValTok{{-}2}\NormalTok{) }\SpecialCharTok{*}\NormalTok{ (n}\DecValTok{{-}3}\NormalTok{))) }\SpecialCharTok{+} \DecValTok{3}
\NormalTok{  \}}
  \FunctionTok{return}\NormalTok{(kurtosis)}
\NormalTok{\}}

\CommentTok{\#fun2(c(1,4,9, 11), TRUE)}
\CommentTok{\#fun2(c(1,4,9, 11), FALSE)}
\end{Highlighting}
\end{Shaded}

\hypertarget{section-9}{%
\subsection{2.3}\label{section-9}}

Download historical price for ticker ''SPY'' for the whole 2012 and 2013
years with quantmod package, use its adjusted close price to calculate
daily log return (Note the adjusted close price is different from the
``adjusted'' for sample moments).

\begin{Shaded}
\begin{Highlighting}[]
\FunctionTok{getSymbols}\NormalTok{(}\AttributeTok{Symbols =} \StringTok{"SPY"}\NormalTok{, }\AttributeTok{from =} \StringTok{"2012{-}01{-}01"}\NormalTok{, }\AttributeTok{to =} \StringTok{\textquotesingle{}2014{-}01{-}01\textquotesingle{}}\NormalTok{)}
\end{Highlighting}
\end{Shaded}

\begin{verbatim}
## [1] "SPY"
\end{verbatim}

\begin{Shaded}
\begin{Highlighting}[]
\NormalTok{SPY }\OtherTok{\textless{}{-}} \FunctionTok{data.frame}\NormalTok{(SPY)}
\FunctionTok{head}\NormalTok{(SPY)}
\end{Highlighting}
\end{Shaded}

\begin{verbatim}
##            SPY.Open SPY.High SPY.Low SPY.Close SPY.Volume SPY.Adjusted
## 2012-01-03   127.76   128.38  127.43    127.50  193697900     103.2023
## 2012-01-04   127.20   127.81  126.71    127.70  127186500     103.3642
## 2012-01-05   127.01   128.23  126.43    128.04  173895000     103.6394
## 2012-01-06   128.20   128.22  127.29    127.71  148050000     103.3723
## 2012-01-09   128.00   128.18  127.41    128.02   99530200     103.6232
## 2012-01-10   129.39   129.65  128.95    129.13  115282000     104.5217
\end{verbatim}

\begin{Shaded}
\begin{Highlighting}[]
\NormalTok{SPY.log.daily.return }\OtherTok{\textless{}{-}} \FunctionTok{diff}\NormalTok{(}\FunctionTok{log}\NormalTok{(SPY}\SpecialCharTok{$}\NormalTok{SPY.Adjusted))}
\end{Highlighting}
\end{Shaded}

\hypertarget{section-10}{%
\subsection{2.4}\label{section-10}}

Calculate the adjusted and unadjusted skewness for the daily log return
in 2.3 using the function you defined. (both numbers should be close to
-0.15)

\begin{Shaded}
\begin{Highlighting}[]
\NormalTok{adj\_skewness }\OtherTok{\textless{}{-}} \FunctionTok{fun}\NormalTok{(SPY.log.daily.return, }\ConstantTok{TRUE}\NormalTok{)}
\NormalTok{adj\_skewness}
\end{Highlighting}
\end{Shaded}

\begin{verbatim}
## [1] -0.1602144
\end{verbatim}

\begin{Shaded}
\begin{Highlighting}[]
\NormalTok{unadj\_skewness }\OtherTok{\textless{}{-}}\FunctionTok{fun}\NormalTok{(SPY.log.daily.return, }\ConstantTok{FALSE}\NormalTok{)}
\NormalTok{unadj\_skewness}
\end{Highlighting}
\end{Shaded}

\begin{verbatim}
## [1] -0.1597343
\end{verbatim}

\hypertarget{section-11}{%
\subsection{2.5}\label{section-11}}

Calculate the adjusted and unadjusted kurtosis for the daily log return
in 2.3 using the function you defined. (both numbers should be close to
4.1)

\begin{Shaded}
\begin{Highlighting}[]
\NormalTok{adj\_kurtosis }\OtherTok{\textless{}{-}} \FunctionTok{fun2}\NormalTok{(SPY.log.daily.return, }\ConstantTok{TRUE}\NormalTok{)}
\NormalTok{adj\_kurtosis}
\end{Highlighting}
\end{Shaded}

\begin{verbatim}
## [1] 4.135122
\end{verbatim}

\begin{Shaded}
\begin{Highlighting}[]
\NormalTok{unadj\_kurtosis }\OtherTok{\textless{}{-}}\FunctionTok{fun2}\NormalTok{(SPY.log.daily.return, }\ConstantTok{FALSE}\NormalTok{)}
\NormalTok{unadj\_kurtosis}
\end{Highlighting}
\end{Shaded}

\begin{verbatim}
## [1] 4.111873
\end{verbatim}

\end{document}
