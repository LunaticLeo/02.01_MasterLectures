% Options for packages loaded elsewhere
\PassOptionsToPackage{unicode}{hyperref}
\PassOptionsToPackage{hyphens}{url}
%
\documentclass[
]{article}
\usepackage{amsmath,amssymb}
\usepackage{lmodern}
\usepackage{iftex}
\ifPDFTeX
  \usepackage[T1]{fontenc}
  \usepackage[utf8]{inputenc}
  \usepackage{textcomp} % provide euro and other symbols
\else % if luatex or xetex
  \usepackage{unicode-math}
  \defaultfontfeatures{Scale=MatchLowercase}
  \defaultfontfeatures[\rmfamily]{Ligatures=TeX,Scale=1}
\fi
% Use upquote if available, for straight quotes in verbatim environments
\IfFileExists{upquote.sty}{\usepackage{upquote}}{}
\IfFileExists{microtype.sty}{% use microtype if available
  \usepackage[]{microtype}
  \UseMicrotypeSet[protrusion]{basicmath} % disable protrusion for tt fonts
}{}
\makeatletter
\@ifundefined{KOMAClassName}{% if non-KOMA class
  \IfFileExists{parskip.sty}{%
    \usepackage{parskip}
  }{% else
    \setlength{\parindent}{0pt}
    \setlength{\parskip}{6pt plus 2pt minus 1pt}}
}{% if KOMA class
  \KOMAoptions{parskip=half}}
\makeatother
\usepackage{xcolor}
\usepackage[margin=1in]{geometry}
\usepackage{color}
\usepackage{fancyvrb}
\newcommand{\VerbBar}{|}
\newcommand{\VERB}{\Verb[commandchars=\\\{\}]}
\DefineVerbatimEnvironment{Highlighting}{Verbatim}{commandchars=\\\{\}}
% Add ',fontsize=\small' for more characters per line
\usepackage{framed}
\definecolor{shadecolor}{RGB}{248,248,248}
\newenvironment{Shaded}{\begin{snugshade}}{\end{snugshade}}
\newcommand{\AlertTok}[1]{\textcolor[rgb]{0.94,0.16,0.16}{#1}}
\newcommand{\AnnotationTok}[1]{\textcolor[rgb]{0.56,0.35,0.01}{\textbf{\textit{#1}}}}
\newcommand{\AttributeTok}[1]{\textcolor[rgb]{0.77,0.63,0.00}{#1}}
\newcommand{\BaseNTok}[1]{\textcolor[rgb]{0.00,0.00,0.81}{#1}}
\newcommand{\BuiltInTok}[1]{#1}
\newcommand{\CharTok}[1]{\textcolor[rgb]{0.31,0.60,0.02}{#1}}
\newcommand{\CommentTok}[1]{\textcolor[rgb]{0.56,0.35,0.01}{\textit{#1}}}
\newcommand{\CommentVarTok}[1]{\textcolor[rgb]{0.56,0.35,0.01}{\textbf{\textit{#1}}}}
\newcommand{\ConstantTok}[1]{\textcolor[rgb]{0.00,0.00,0.00}{#1}}
\newcommand{\ControlFlowTok}[1]{\textcolor[rgb]{0.13,0.29,0.53}{\textbf{#1}}}
\newcommand{\DataTypeTok}[1]{\textcolor[rgb]{0.13,0.29,0.53}{#1}}
\newcommand{\DecValTok}[1]{\textcolor[rgb]{0.00,0.00,0.81}{#1}}
\newcommand{\DocumentationTok}[1]{\textcolor[rgb]{0.56,0.35,0.01}{\textbf{\textit{#1}}}}
\newcommand{\ErrorTok}[1]{\textcolor[rgb]{0.64,0.00,0.00}{\textbf{#1}}}
\newcommand{\ExtensionTok}[1]{#1}
\newcommand{\FloatTok}[1]{\textcolor[rgb]{0.00,0.00,0.81}{#1}}
\newcommand{\FunctionTok}[1]{\textcolor[rgb]{0.00,0.00,0.00}{#1}}
\newcommand{\ImportTok}[1]{#1}
\newcommand{\InformationTok}[1]{\textcolor[rgb]{0.56,0.35,0.01}{\textbf{\textit{#1}}}}
\newcommand{\KeywordTok}[1]{\textcolor[rgb]{0.13,0.29,0.53}{\textbf{#1}}}
\newcommand{\NormalTok}[1]{#1}
\newcommand{\OperatorTok}[1]{\textcolor[rgb]{0.81,0.36,0.00}{\textbf{#1}}}
\newcommand{\OtherTok}[1]{\textcolor[rgb]{0.56,0.35,0.01}{#1}}
\newcommand{\PreprocessorTok}[1]{\textcolor[rgb]{0.56,0.35,0.01}{\textit{#1}}}
\newcommand{\RegionMarkerTok}[1]{#1}
\newcommand{\SpecialCharTok}[1]{\textcolor[rgb]{0.00,0.00,0.00}{#1}}
\newcommand{\SpecialStringTok}[1]{\textcolor[rgb]{0.31,0.60,0.02}{#1}}
\newcommand{\StringTok}[1]{\textcolor[rgb]{0.31,0.60,0.02}{#1}}
\newcommand{\VariableTok}[1]{\textcolor[rgb]{0.00,0.00,0.00}{#1}}
\newcommand{\VerbatimStringTok}[1]{\textcolor[rgb]{0.31,0.60,0.02}{#1}}
\newcommand{\WarningTok}[1]{\textcolor[rgb]{0.56,0.35,0.01}{\textbf{\textit{#1}}}}
\usepackage{graphicx}
\makeatletter
\def\maxwidth{\ifdim\Gin@nat@width>\linewidth\linewidth\else\Gin@nat@width\fi}
\def\maxheight{\ifdim\Gin@nat@height>\textheight\textheight\else\Gin@nat@height\fi}
\makeatother
% Scale images if necessary, so that they will not overflow the page
% margins by default, and it is still possible to overwrite the defaults
% using explicit options in \includegraphics[width, height, ...]{}
\setkeys{Gin}{width=\maxwidth,height=\maxheight,keepaspectratio}
% Set default figure placement to htbp
\makeatletter
\def\fps@figure{htbp}
\makeatother
\setlength{\emergencystretch}{3em} % prevent overfull lines
\providecommand{\tightlist}{%
  \setlength{\itemsep}{0pt}\setlength{\parskip}{0pt}}
\setcounter{secnumdepth}{-\maxdimen} % remove section numbering
\ifLuaTeX
  \usepackage{selnolig}  % disable illegal ligatures
\fi
\IfFileExists{bookmark.sty}{\usepackage{bookmark}}{\usepackage{hyperref}}
\IfFileExists{xurl.sty}{\usepackage{xurl}}{} % add URL line breaks if available
\urlstyle{same} % disable monospaced font for URLs
\hypersetup{
  pdftitle={FE515 2022A Assignment 3},
  pdfauthor={Yufu Liao},
  hidelinks,
  pdfcreator={LaTeX via pandoc}}

\title{FE515 2022A Assignment 3}
\author{Yufu Liao}
\date{04/21/2023}

\begin{document}
\maketitle

\hypertarget{question-1-50-points}{%
\section{Question 1: (50 points)}\label{question-1-50-points}}

\hypertarget{section}{%
\subsection{1.1}\label{section}}

Download option prices of ticker \^{}VIX for all expiration dates and
name it VIX.options

\begin{Shaded}
\begin{Highlighting}[]
\FunctionTok{library}\NormalTok{(quantmod)}
\end{Highlighting}
\end{Shaded}

\begin{verbatim}
## Warning: package 'quantmod' was built under R version 4.2.3
\end{verbatim}

\begin{verbatim}
## Loading required package: xts
\end{verbatim}

\begin{verbatim}
## Warning: package 'xts' was built under R version 4.2.3
\end{verbatim}

\begin{verbatim}
## Loading required package: zoo
\end{verbatim}

\begin{verbatim}
## Warning: package 'zoo' was built under R version 4.2.3
\end{verbatim}

\begin{verbatim}
## 
## Attaching package: 'zoo'
\end{verbatim}

\begin{verbatim}
## The following objects are masked from 'package:base':
## 
##     as.Date, as.Date.numeric
\end{verbatim}

\begin{verbatim}
## Loading required package: TTR
\end{verbatim}

\begin{verbatim}
## Warning: package 'TTR' was built under R version 4.2.3
\end{verbatim}

\begin{verbatim}
## Registered S3 method overwritten by 'quantmod':
##   method            from
##   as.zoo.data.frame zoo
\end{verbatim}

\begin{Shaded}
\begin{Highlighting}[]
\FunctionTok{library}\NormalTok{(xts)}

\NormalTok{VIX.options }\OtherTok{\textless{}{-}} \FunctionTok{getOptionChain}\NormalTok{(}\StringTok{"\^{}VIX"}\NormalTok{, }\ConstantTok{NULL}\NormalTok{)}
\end{Highlighting}
\end{Shaded}

\hypertarget{section-1}{%
\subsection{1.2}\label{section-1}}

Download the current price (last quote price) for \^{}VIX

\begin{Shaded}
\begin{Highlighting}[]
\NormalTok{(VIX.current.price }\OtherTok{\textless{}{-}} \FunctionTok{getQuote}\NormalTok{(}\StringTok{"\^{}VIX"}\NormalTok{)}\SpecialCharTok{$}\NormalTok{Last)}
\end{Highlighting}
\end{Shaded}

\begin{verbatim}
## [1] 16.76
\end{verbatim}

\hypertarget{section-2}{%
\subsection{1.3}\label{section-2}}

For calls and puts of VIX.options at each expiration calculate the
average of Bid and Ask. Create a new column named `Price' to contain the
result.

\begin{Shaded}
\begin{Highlighting}[]
\ControlFlowTok{for}\NormalTok{ (i }\ControlFlowTok{in} \DecValTok{1}\SpecialCharTok{:}\FunctionTok{length}\NormalTok{(VIX.options)) \{}
\NormalTok{  VIX.options[[i]]}\SpecialCharTok{$}\NormalTok{calls}\SpecialCharTok{$}\NormalTok{Price }\OtherTok{\textless{}{-}}\NormalTok{ (VIX.options[[i]]}\SpecialCharTok{$}\NormalTok{calls}\SpecialCharTok{$}\NormalTok{Bid }\SpecialCharTok{+}\NormalTok{ VIX.options[[i]]}\SpecialCharTok{$}\NormalTok{calls}\SpecialCharTok{$}\NormalTok{Ask) }\SpecialCharTok{*} \FloatTok{0.5}
\NormalTok{  VIX.options[[i]]}\SpecialCharTok{$}\NormalTok{puts}\SpecialCharTok{$}\NormalTok{Price }\OtherTok{\textless{}{-}}\NormalTok{ (VIX.options[[i]]}\SpecialCharTok{$}\NormalTok{puts}\SpecialCharTok{$}\NormalTok{Bid }\SpecialCharTok{+}\NormalTok{ VIX.options[[i]]}\SpecialCharTok{$}\NormalTok{puts}\SpecialCharTok{$}\NormalTok{Ask) }\SpecialCharTok{*} \FloatTok{0.5}
\NormalTok{\}}
\end{Highlighting}
\end{Shaded}

\hypertarget{section-3}{%
\subsection{1.4}\label{section-3}}

For calls and puts of VIX.options at each expiration, add a column of
InTheMoney, which takes value TRUE when it is in-the-money, and FALSE
otherwise. Compare it to ITM column to check your results. (Hint. A call
option is in-the-money when its strike is less than the current price of
underlying. A put option is in-the-money if its strike is greater than
the current price of underlying. And the current price of underlying is
the last quote price from 1.2)

\begin{Shaded}
\begin{Highlighting}[]
\ControlFlowTok{for}\NormalTok{ (i }\ControlFlowTok{in} \DecValTok{1}\SpecialCharTok{:}\FunctionTok{length}\NormalTok{(VIX.options)) \{}
\NormalTok{  VIX.options[[i]]}\SpecialCharTok{$}\NormalTok{calls}\SpecialCharTok{$}\NormalTok{InTheMoney }\OtherTok{\textless{}{-}} \FunctionTok{ifelse}\NormalTok{(VIX.options[[i]]}\SpecialCharTok{$}\NormalTok{calls}\SpecialCharTok{$}\NormalTok{Strike }\SpecialCharTok{\textless{}}\NormalTok{ VIX.current.price, }\ConstantTok{TRUE}\NormalTok{, }\ConstantTok{FALSE}\NormalTok{)}
\NormalTok{  VIX.options[[i]]}\SpecialCharTok{$}\NormalTok{puts}\SpecialCharTok{$}\NormalTok{InTheMoney }\OtherTok{\textless{}{-}} \FunctionTok{ifelse}\NormalTok{(VIX.options[[i]]}\SpecialCharTok{$}\NormalTok{puts}\SpecialCharTok{$}\NormalTok{Strike }\SpecialCharTok{\textless{}}\NormalTok{ VIX.current.price, }\ConstantTok{TRUE}\NormalTok{, }\ConstantTok{FALSE}\NormalTok{)}
\NormalTok{\}}
\end{Highlighting}
\end{Shaded}

\hypertarget{section-4}{%
\subsection{1.5}\label{section-4}}

For calls and puts of VIX at each expiration, delete all the fields
except Strike, Bid, Ask, Price, and In-The-Money, and save them in .csv
files with the format ``VIXdata2021-09- 26Exp2021-10-08puts.csv'', here
2021-09-26 should be replaced by the date you download the data, and
2021-10-08 should be replaced by the date of expiration.

\begin{Shaded}
\begin{Highlighting}[]
\NormalTok{ex }\OtherTok{\textless{}{-}} \FunctionTok{names}\NormalTok{(VIX.options)}
\ControlFlowTok{for}\NormalTok{ (i }\ControlFlowTok{in} \DecValTok{1}\SpecialCharTok{:}\FunctionTok{length}\NormalTok{(VIX.options)) \{}
\NormalTok{  VIX.options[[i]]}\SpecialCharTok{$}\NormalTok{calls }\OtherTok{\textless{}{-}}\NormalTok{ VIX.options[[i]]}\SpecialCharTok{$}\NormalTok{calls[}\FunctionTok{c}\NormalTok{(}\StringTok{"Strike"}\NormalTok{, }\StringTok{"Bid"}\NormalTok{, }\StringTok{"Ask"}\NormalTok{, }\StringTok{"Price"}\NormalTok{, }\StringTok{"InTheMoney"}\NormalTok{)]}
\NormalTok{  VIX.options[[i]]}\SpecialCharTok{$}\NormalTok{puts }\OtherTok{\textless{}{-}}\NormalTok{ VIX.options[[i]]}\SpecialCharTok{$}\NormalTok{puts[}\FunctionTok{c}\NormalTok{(}\StringTok{"Strike"}\NormalTok{, }\StringTok{"Bid"}\NormalTok{, }\StringTok{"Ask"}\NormalTok{, }\StringTok{"Price"}\NormalTok{, }\StringTok{"InTheMoney"}\NormalTok{)]}
  \FunctionTok{write.csv}\NormalTok{(VIX.options[[i]]}\SpecialCharTok{$}\NormalTok{puts, }\AttributeTok{file =} \FunctionTok{paste}\NormalTok{(}\StringTok{"VIXdata"}\NormalTok{, }\FunctionTok{Sys.Date}\NormalTok{(), }\StringTok{"Exp"}\NormalTok{, ex[i], }\StringTok{"puts.csv"}\NormalTok{, }\AttributeTok{sep =} \StringTok{""}\NormalTok{))}
\NormalTok{\}}
\NormalTok{ex}
\end{Highlighting}
\end{Shaded}

\begin{verbatim}
##  [1] "Apr.26.2023" "May.03.2023" "May.10.2023" "May.17.2023" "May.24.2023"
##  [6] "Jun.21.2023" "Jul.19.2023" "Aug.16.2023" "Sep.20.2023" "Oct.18.2023"
## [11] "Nov.15.2023" "Dec.20.2023"
\end{verbatim}

\hypertarget{question-2}{%
\section{Question 2}\label{question-2}}

\hypertarget{section-5}{%
\subsection{2.1}\label{section-5}}

Using Monte-Carlo Simulation to estimate the put option price using S0 =
100, K = 100, T = 1, σ = 0.2, r = 0.05, you can use number of steps n =
252 and number of paths m = 10000

\begin{Shaded}
\begin{Highlighting}[]
\NormalTok{S0 }\OtherTok{\textless{}{-}} \DecValTok{100}
\NormalTok{K }\OtherTok{\textless{}{-}} \DecValTok{100}
\NormalTok{T1 }\OtherTok{\textless{}{-}} \DecValTok{1}
\NormalTok{sigma }\OtherTok{\textless{}{-}} \FloatTok{0.2}
\NormalTok{r }\OtherTok{\textless{}{-}} \FloatTok{0.05}
\NormalTok{func.mc }\OtherTok{\textless{}{-}} \ControlFlowTok{function}\NormalTok{(S0, K, T1, sigma, r) \{}
\NormalTok{  n }\OtherTok{=} \DecValTok{252}
\NormalTok{  m }\OtherTok{=} \DecValTok{10000}
\NormalTok{  h }\OtherTok{\textless{}{-}}\NormalTok{ T1 }\SpecialCharTok{/}\NormalTok{ n}
\NormalTok{  S.vec }\OtherTok{\textless{}{-}} \FunctionTok{rep}\NormalTok{(S0, m)}
\NormalTok{  Z }\OtherTok{\textless{}{-}} \FunctionTok{matrix}\NormalTok{(}\FunctionTok{rnorm}\NormalTok{(n }\SpecialCharTok{*}\NormalTok{ m), }\AttributeTok{nrow =}\NormalTok{ n)}
  \ControlFlowTok{for}\NormalTok{ (i }\ControlFlowTok{in} \DecValTok{1}\SpecialCharTok{:}\NormalTok{n) \{}
\NormalTok{  S.vec }\OtherTok{\textless{}{-}}\NormalTok{ S.vec }\SpecialCharTok{+}\NormalTok{ r }\SpecialCharTok{*}\NormalTok{ S.vec }\SpecialCharTok{*}\NormalTok{ h }\SpecialCharTok{+}\NormalTok{ sigma }\SpecialCharTok{*}\NormalTok{ S.vec }\SpecialCharTok{*}\NormalTok{ Z[i,] }\SpecialCharTok{*} \FunctionTok{sqrt}\NormalTok{(h)}
\NormalTok{  \}}
  \FunctionTok{return}\NormalTok{(}\FunctionTok{exp}\NormalTok{(}\SpecialCharTok{{-}}\NormalTok{r }\SpecialCharTok{*}\NormalTok{ T1) }\SpecialCharTok{*} \FunctionTok{mean}\NormalTok{(}\FunctionTok{pmax}\NormalTok{(}\DecValTok{100} \SpecialCharTok{{-}}\NormalTok{ S.vec, }\DecValTok{0}\NormalTok{)))}
\NormalTok{\}}

\FunctionTok{func.mc}\NormalTok{(S0, K, T1, sigma, r)}
\end{Highlighting}
\end{Shaded}

\begin{verbatim}
## [1] 5.67446
\end{verbatim}

\hypertarget{section-6}{%
\subsection{2.2}\label{section-6}}

Implement Black-Scholes formula for pricing the put option

\begin{Shaded}
\begin{Highlighting}[]
\NormalTok{func.bs }\OtherTok{\textless{}{-}} \ControlFlowTok{function}\NormalTok{(S0, K, T1, sigma, r) \{}
\NormalTok{  d1 }\OtherTok{\textless{}{-}}\NormalTok{ (}\FunctionTok{log}\NormalTok{(S0 }\SpecialCharTok{/}\NormalTok{ K) }\SpecialCharTok{+}\NormalTok{ (r }\SpecialCharTok{+} \FloatTok{0.5} \SpecialCharTok{*}\NormalTok{ sigma }\SpecialCharTok{\^{}} \DecValTok{2}\NormalTok{) }\SpecialCharTok{*}\NormalTok{ T1) }\SpecialCharTok{/}\NormalTok{ (sigma }\SpecialCharTok{*} \FunctionTok{sqrt}\NormalTok{(T1))}
\NormalTok{  d2 }\OtherTok{\textless{}{-}}\NormalTok{ d1 }\SpecialCharTok{{-}}\NormalTok{ sigma }\SpecialCharTok{*} \FunctionTok{sqrt}\NormalTok{(T1)}
  \FunctionTok{return}\NormalTok{ (}\SpecialCharTok{{-}}\NormalTok{S0 }\SpecialCharTok{*} \FunctionTok{pnorm}\NormalTok{(}\SpecialCharTok{{-}}\NormalTok{d1) }\SpecialCharTok{+} \FunctionTok{exp}\NormalTok{(}\SpecialCharTok{{-}}\NormalTok{r }\SpecialCharTok{*}\NormalTok{ T1) }\SpecialCharTok{*}\NormalTok{ K }\SpecialCharTok{*} \FunctionTok{pnorm}\NormalTok{(}\SpecialCharTok{{-}}\NormalTok{d2))}
\NormalTok{\}}
\FunctionTok{func.bs}\NormalTok{(S0, K, T1, sigma, r)}
\end{Highlighting}
\end{Shaded}

\begin{verbatim}
## [1] 5.573526
\end{verbatim}

\end{document}
